\documentclass{article}
\usepackage[utf8]{inputenc}
\usepackage{amsmath}

\title{SM4 Reversibility Proof}
\author{Anciety}
\date{October 2019}

\begin{document}
\maketitle

\section{Terms And Notes}

The terms and notes used in this proof is as follows.\\
Given a 128-bit plaintext input, divides into word size $(X_0, X_1, X_2, X_3)$, each of $X_i$ is a word (32-bit).\\
The output as well, described as $(Y_0, Y_1, Y_2, Y_3)$, each 32-bit.\\
Round key are denoted as $rk_i$ where $i$ is the ranges $i = 0, 1, .., 31$.\\
$F$ is the round function maps 5 words to a new word. $R$ is a reverse transformation defined as follows:\\
\begin{equation}
    R(a, b, c, d) = (d, c, b, a)
\end{equation}

\section{Structure}

The structure of the SM4 encryption consists a unbalanced Feistel network. Each round, round function is applied to the 4 words and the round key which produces a new word, next round will happen on last 3 words of last round with new word appended.\\
Formally, for round $k$, we have\\
\begin{equation}
\begin{aligned}
    input_k &= (X_{k-1}, X_k, X_{k+1}, X_{k+2})\ for\ 0\leq k \leq 31\\
    output_k &= (X_k, X_{k + 1}, X_{k + 2}, F(X_{k-1}, X_k, X_{k + 1}, X_{k + 2}, rk_{k}))\ for\ 0\leq k \leq 31\\
    input_{k+1} &= output_k\ for\ 0\leq k \leq 30\\
    ciphertext &= R(output_{31})
\end{aligned}
\end{equation}

This gives the overall structure of SM4 encryption.

\section{Proof}
The decryption procedure of SM4 is given as use the same procedure as encryption, except reverse the round key sequence.

Formally, this can be described as, for round $k$, using the same sub-labels as the encryption, we have:\\
\begin{equation}
    \begin{aligned}
    input_k &= (X_{35-k}, X_{34-k}, X_{33-k}, X_{32-k})\\
    output_k &= (X_{34-k}, X_{33-k}, X_{32-k}, F(X_{35-k}, X_{34-k}, X_{33-k}, X_{32-k}, rk_{31-k}))\\
    input_{k+1} &= output_k\ for\ 0 \leq k \leq 30\\
    plaintext = output_{31}
    \end{aligned}
\end{equation}

The order of the input is reversed by the $R$ function accordingly.

Round Function itself is defined as:\\
\begin{equation}
    F(X_0, X_1, X_2, X_3, rk) = X_0 \bigoplus T(X_1 \bigoplus X_2 \bigoplus X_3 \bigoplus rk)
\end{equation}
To prove the decryption is actually correct, we can prove for a single round. If each single round reverses the original procedure correspondingly, it can be proved then.
This is to say, we need to prove that indeed each $X_i$ is the one used exactly in encryption, so to speak.
This can be proved as follows by induction.
Initial status, we have:
\begin{equation}
    (Y_0, Y_1, Y_2, Y_3) = (X_{32}, X_{33}, X_{34}, X_{35}) = R(X_{35}, X_{34}, X_{33}, X_{32})
\end{equation}
This proves when $k = 0$, $input_0$ of decryption input is well-defined.
Next we prove for each round, when $input_i$ is well defined. This is to prove:
\begin{equation}
    F(X_{35-k}, X_{34-k}, X_{33-k}, X_{32-k}, rk_{31-k}) = X_{31-k}
\end{equation}

Then:

\begin{equation*}
    \begin{gathered}
F(X_{35-k}, X_{34-k}, X_{33-k}, X_{32-k}) &= X_{35-k} \bigoplus T(X_{34-k} \bigoplus X_{33-k} \bigoplus X_{32-k}) (by\ definition)\\
&= X_{31-k} \bigoplus T(X_{32-k} \bigoplus X_{33-k} \bigoplus X_{34-k}) \\
\bigoplus T(X_{34-k} \bigoplus X_{33-k} \bigoplus X_{32-k}) (substitute\ with\ encryption\ round)\\
&= X_{31-k}
    \end{gathered}
\end{equation*}

Thus, the decryption procedure is well-defined, which means the decryption can decrypt the ciphertext to plaintext.

\end{document}
